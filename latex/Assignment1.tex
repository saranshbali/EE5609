\documentclass{article}
\usepackage{amsmath,amssymb,amsthm}
\usepackage{enumerate}
\newtheorem{theorem}{Theorem}
\newtheorem{question}[theorem]{Question}
\newtheorem{answer}[theorem]{Solution}
\begin{document}
\begin{question}
	Show that the vectors $\begin{pmatrix} 
		 +2\\-1\\+1 
	\end{pmatrix}$, $\begin{pmatrix} 
	 +1\\-3\\-5 
\end{pmatrix}$ and $\begin{pmatrix} 
+3\\-4\\-4 
\end{pmatrix}$ form the vertices of a right angled trianle.
\end{question}
Solution. Let $v1$, $v2$ and $v3$ be given vectors such that $v1 =\begin{pmatrix} 
	+2\\-1\\+1 
\end{pmatrix}$,\\     
$v2= \begin{pmatrix} 
+1\\-3\\-5 
\end{pmatrix}$ and $v3= \begin{pmatrix} 
+3\\-4\\-4 
\end{pmatrix}$.\\
To show that $v1$, $v2$ and $v3$ form the vertices of a right anled triangle. First we need to show that $v1$, $v2$ and $v3$ are indeed vertices of a triangle.\\
For this we need to see if the vertices satisfy trianle inequality. Let $a$, $b$ and $c$ denote the length of vertices $v1-v2$, $v2-v3$ and $v3-v1$. Now,\\
$a=\sqrt{41}$, $b=\sqrt{6}$ and $c=\sqrt{35}$. We can see that  $a+b>c$, 
$a+c>b$ and $b+c >a$. Thus, the given vertices $v1$, $v2$ and $v3$ form the vertices of a triangle.\\
To show they form right triangle, we need to show that any two sides from given vertices are perpendicular to each other, that means we need to show two of the three given vectors are perprndicular to each other. For this we need to calculate the dot product of the three vectors. Clearly, we can see that $v1.v2 =0$, that means $v1$ is perpendicular to $v2$. Thus, the vectors $v1$, $v2$ and $v3$ form vertices of a right angled trianle.  




\end{document}
